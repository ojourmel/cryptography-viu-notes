% Oliver Jourmel
% 567552674
%
%
% MATH 331 Cryptography Course
% Spring 2015
%
\documentclass[letterpaper,10pt,twoside]{report}

% Document margins
\topmargin=-2cm
\textheight=24cm
\textwidth=16cm
\oddsidemargin=0cm
\evensidemargin=0cm

\setlength{\parindent}{0mm}

% Packages
\usepackage{booktabs}
\usepackage{pifont}
\usepackage{graphicx}
\usepackage{psfrag}
\usepackage{color}
\usepackage{colortbl}
\usepackage{fancyhdr}
\usepackage{parskip}
\usepackage{latexsym}
\usepackage{enumerate}
\usepackage{amssymb}
\usepackage{amsmath}
\usepackage{amsthm}
\usepackage{bm}
\usepackage{cancel}
\usepackage{appendix}
\usepackage{array}
\usepackage[bf]{titlesec}

\usepackage{todonotes}

% Heading Formats
\titleformat{\chapter}[display]
{\sffamily\huge\bfseries}{Chapter \thechapter}{1ex}{}

\titleformat{\section}
{\sffamily\large\bfseries}{\thesection}{1em}{}

\titleformat{\subsection}
{\sffamily\normalsize\bfseries}{\thesubsection}{1em}{}

\titleformat{\subsubsection}
{\sffamily\normalsize\bfseries}{\thesubsubsection}{1em}{}


% Custom Environments, Headings, and Symbols
\newcounter{example}
\newenvironment{example}
{\underline{\sffamily\bfseries Example \theexample}\\}
{\hspace*{\fill}$\bullet$ \stepcounter{example}}

\theoremstyle{plain}
\newtheorem{theorem}{Theorem}[subsection]
\newtheorem{corollary}{Corollary}[theorem]
\newtheorem{lemma}[theorem]{Lemma}

\theoremstyle{definition}
\newtheorem{definition}{Definition}[section]

\theoremstyle{remark}
\newtheorem*{fact}{Fact}
\newtheorem*{remark}{Remark}

\newcommand{\len}{\mathrm{len}}


%%%%%%%%%%%%%%%%%%%%%%%%%%%%%%%%%%%%%%%%%%%%%%%%%%%%%%%%%%%%%%%%%%%%%%%%%%%%%%

\begin{document}

\pagestyle{fancy}

\renewcommand{\chaptermark}[1]{\markboth{\chaptername \ \thechapter.\ #1}{}} 
\renewcommand{\sectionmark}[1]{\markright{\thesection.\ #1}{}}
\fancyhead[LE,RO]{\sffamily\bfseries \rightmark}
\fancyhead[LO,RE]{\sffamily\bfseries \leftmark}

\setcounter{chapter}{0}
\setcounter{definition}{1}
\setcounter{example}{1}

\chapter{Cryptography and Coding Theory}
\section{Introduction}
\subsection{Terminology}

\begin{definition}{What is Crypt*}

\begin{tabular}{ll}
   Cryptology      & $ = $ the study of sending information over secure channels \\
   Cryptography    & $ = $ methods used to design such \emph{Cryptosystems} \\
   Cryptanalysis   & $ = $ methods used to attack or break \emph{Cryptosystems} \\
   Cryptology      & $ = $ Cryptography $ + $ Cryptanalysis \\
   & \\
   Coding Theory   & $ = $ the study of replacing a set of input symbols \\
                   & $ = $ study of sending information over noisy channels \\
\end{tabular}

\end{definition}

\begin{definition}{Dramatis Personae}

\begin{tabular}{ll}
   Alice     & $ = $ a person who wishes to communicate with Bob. \\
   Bob       & $ = $ a person who wishes to communicate with Alice. Bob is sometimes called Bill. \\
   Eve       & $ = $ an easvesdropper; someone who wishes to intercept and decrypt message between Alice and Bob. \\
   Oscar     & $ = $ a passive version of Eve. \\
   Mallary   & $ = $ a malicious version of Eve. \\
\end{tabular}

\begin{fact}
   The original message is referred to as \emph{plaintext}.
\end{fact}

\begin{fact}
   The original message is referred to as \emph{ciphertext}.
\end{fact}

\todo{Figure}
\end{definition}

\begin{definition}{Kerckhoff's Principal (1833 ``La Cryptograph Militaire'')} \\
``In assessing the security of a cryptosystems, one should always assume the enemy knows the method of encryption and decryption being used. The security must be based on a key, rather than the obscurity of the system.''
\end{definition}

Eve could potentially have one of the following goals:\\
%\vspace{-1em}
\begin{itemize}
   \item{She wants read a message.}
   \item{She wants to be able to read all messages sent with a certain key.}
   \item{She wants to corrupt the message in such a way that Bob think he has received the message from Alice.}
   \item{She wants to intercept all messages from Alice and perform man-in-the-middle-attack.}
\end{itemize}

\subsection{Cryptographic Attacks}
\begin{definition}{\ } \\
   There are four main types of cryptographic attacks:
\begin{enumerate}
   \item{Ciphertext only attack: \\ Eve has gained access to some ciphertext and uses only this information to determine the plaintext.}
   \item{Known plaintext attack: \\ Eve has gained access to some plaintext and the corresponding ciphertext. She uses this to find the key, and decrypt all future messages sent with that key.}
   \item{Chosen plaintext attack: \\ Eve temporarily gains access to an encryption machine and chooses some plaintext, generating the corresponding ciphertext.}
   \item{Chosen ciphertext attack: \\ Eve temporarily gains access to a decryption machine and chooses some ciphertext, generating the corresponding plaintext.}
\end{enumerate}
\end{definition}

\subsection{Keys}
\begin{definition}{\ } \\
   There are two main types of keys:
\begin{enumerate}
   \item{Symmetric Keys: \\ Alice and Bob each have both the encryption and decryption keys.}
   \item{Public Key Cryptography: \\ Bob reviles the encryption key to Alice and anyone else. Only Bob would have the decryption key.}
\end{enumerate}
\end{definition}

\section{Historical Ciphers}
\subsection{Cesar Cipher (Substitution Cipher)}

\begin{example}

\begin{tabular}{c|*{26}{c@{\hspace{2.75mm}}}}
   Plain  & A & B & C & D & E & F & G & H & I & J & K & L & M & N & O & P & Q & R & S & T & U & V & W & X & Y & Z \\
   Cipher & S & T & U & V & W & X & Y & Z & A & B & C & D & E & F & G & H & I & J & K & L & M & N & O & P & Q & R \\
\end{tabular}

\begin{fact}
   Shift the alphabet by 8 slots: 8 is the key.
\end{fact}

\[\texttt{HI ERICA} \rightarrow \texttt{za wjaus} \rightarrow \texttt{zawjaus}\]

\end{example}

Leaving spaces in the cipher text leaves a clue as to the word structure. Similarly, punctuation marks should be left out.

\subsection{The Beale Cipher}
2ed cipher was cracked as a book cipher

a book code
\todo{insert example beale cipher}


\subsection{Zodiac 340}

serial killer write letters to police encrypted
one was cracked using a substitution cipher

\end{document}

